\section{Initial Specifications}

The Union Cycliste Internationale has requested our IT services to develop an Information System to manage the riders belonging to the various teams of the federation based on the following specifications:

"Riders are individuals for whom we record their first and last name, height, date of birth, and the team they belong to. A team is identified by its name, has a budget, and a sports director who is a person with a known first and last name and date of birth. Teams are funded by sponsors, which may vary each year, and for each sponsor, we record their name, address, and field of activity. 

A race corresponds to a race name (e.g., ‘Tour de France’), and we know the total distance to be covered. It can include one or more stages, each identified by its sequence number (e.g., ‘Stage 3’), date, type (e.g., ‘Individual Time Trial’), starting city, and finishing city. 

For each rider who participated in a stage of a race, we record the ranking they achieved in that stage. For each race, we track the final winner and the team they belong to. 

For every race, teams employ soigneurs, who are individuals for whom we store their first and last name, date of birth, and nationality. Additionally, at each stage, we record which doses of which product(s) a soigneur administered to a rider. 

A product is identified by a product number, has a name, an indication (e.g., ‘muscle pain’), a contraindication (e.g., ‘do not administer to individuals under 20 years old’), and a dosage recommendation (e.g., ‘1 tablet per day’). 

In this production database, only current information (related to the ongoing edition) about the race, riders, teams, etc., is stored."
